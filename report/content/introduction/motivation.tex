\section{Motivation}
\label{sec:motivation}

So, we have that CPUs are general purpose processing units that perform very well on their own, but they are expensive in power.
On the other hand, we have the GPU which is more simple and built for heavy computations.
It is more power efficient, but the programming model is more restrictive.
The developer must now think in parallel and make sure that the data is distributed in a proper fashion.

Put another way, the CPU is built for optimising latency, and the GPU is built for optimising throughput.
Latency describes how fast something was performed, i.e. how many seconds did the job take before finishing.
Throughput describes how many jobs were performed within a time frame, i.e. how many jobs per hour were performed~\cite{farber2011cuda}.

Beyond our personal interest in parallel computing, the modern information technology society is being faced with handling very large amounts of data.
The buzzword is Big Data and with a lot of data the GPU gets very interesting, because it can crunch through a lot of data very quickly.
Applications such as machine learning, numerical analysis, and graphics are well suited to be solved on a GPU because the problem's nature is easily made parallel~\cite{amdahlorgustafson2011,chen2014data}.
