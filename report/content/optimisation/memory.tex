\section{Memory Optimisations}
\label{sec:memory optimisations}
\subsection{Coelesced Memory Access}
\label{sec:coalesced}

In \cref{sec:GPGPU memory architecture} we introduced the L1 cache design being computed for spatial locality.

Spartial locality promotes the use of ``coalesced'' memory access by the individual threads.
Coalesced memory access means to have threads access successive memory location.
Since quering the global memory for a data value will often cause a memory transfer larger than what the thread requires and place the data in the faster cache on the SM.
When the sebsequent thread queries for the successive value it will be returned from the L1 cache instead of the global memory, giving a substantial speed-up.

In order to utilize coalesced the memory layout of the data needs to be in a structure that promotes data to be successive rather than strided.\cite{udacity}
This leads to two different approached for storing data, Array of Structures and Structures of Arrays, examplified with RGB images in \cref{lst:soa aos}.

\begin{lstlisting}[caption={Example of SoA and AoS with RGB images}, label={lst:soa aos}]
struct {
  int R, G, B;
} AoS[N];

struct {
  int R[N], G[N], B[N];
} SoA;
\end{lstlisting}

The Array of Structures lays out the images accordingly : $R[1], G[1], B[1], R[2] G[2], B[2] ... R[N], G[N], B[N]$, whereas the Structure of Arrays would layout the images in the following manner : $R[1], R[2], ... R[N], G[1], G[2], ... G[N], B[1], B[2], ... B[N]$.
The Structures of Arrays thus manages to layout the data in a coalsced way, where as the Arrays of Structures lays out the data as strided.

\subsection{Shared Memory}
In \cref{sec:grids blocks threads} we introduced blocks as having a set amount of shared memory available.

Shared memory is a defined space of memory in the SM's L1 cache that a set of threads within a block can read/write to during block execution.
The memory will disapear when the block terminates.
In this report we will consider two types of instances where shared memory could give a beneficial speed-up.

The first is when a set of threads have to perform several read/write operations to a fixed set of global memory, which we as use in \cref{sec:reduce}.
By mapping the global memory to shared memory, computing the set of operations in shared memory and then writing back to global memory allows us utilize the fast access to shared memory.

The second is to coalesce writes by collecting them in shared memory.
This type of coalescing is especially interesting in the the transpose problem where data is read row-wise but written column-wise. In \cref{sec:transpose} will introduce how we used tiles in shared memory to allow coalesced writing.
