\section{General Multi-core CPU Architecture}
\label{sec:cpu}

We will consider the architecture of a general multi-core Central Processing Unit (CPU) and what memory it can communicate with.
A multi-core CPU has two or more independent processing cores, which are Multiple Instructions Multiple Data (MIMD) cores.
This means that they can work on different instructions with different data simultaneously.

These cores have at least one layer of private cached memory.
Furthermore, there is at least one layer of public memory, where the largest is called Random Access Memory (RAM), and it thus shared memory between all the cores.
The further down the layered hierarchy the given memory is, the slower it is for the CPU to receive data from it.
