\section{Coalesced Memory Access}
\label{sec:coalesced}

In \cref{sec:GPGPU memory architecture} we introduced the L1 cache design being computed for spatial locality.

Spartial locality promotes the use of ``coalesced'' memory access by the individual threads.
Coalesced memory access means to have threads access successive memory location.
Since quering the global memory for a data value will often cause a memory transfer larger than what the thread requires and place the data in the faster cache on the SM.
When the sebsequent thread queries for the successive value it will be returned from the L1 cache instead of the global memory, giving a substantial speed-up.

In order to utilize coalesced the memory layout of the data needs to be in a structure that promotes data to be successive rather than strided.\cite{udacity}
This leads to two different approached for storing data, Array of Structures and Structures of Arrays, examplified with RGB images in \cref{lst:soa aos}.

\begin{lstlisting}[caption={Example of SoA and AoS with RGB images}, label={lst:soa aos}]
struct {
  int R, G, B;
} AoS[N];

struct {
  int R[N], G[N], B[N];
} SoA;
\end{lstlisting}

The Array of Structures lays out the images accordingly : $R[1], G[1], B[1], R[2] G[2], B[2] ... R[N], G[N], B[N]$, whereas the Structure of Arrays would layout the images in the following manner : $R[1], R[2], ... R[N], G[1], G[2], ... G[N], B[1], B[2], ... B[N]$.
The Structures of Arrays thus manages to layout the data in a coalsced way, where as the Arrays of Structures lays out the data as strided.
