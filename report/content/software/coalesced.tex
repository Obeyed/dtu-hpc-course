\section{Coalesced Memory Access}
\label{sec:coalesced}

In \cref{sec:cuda memory architecture} we introduced that the L1 cache is design to work in a spatial locality, where the neighbouring data is assumed to be used in the near future.
Data can be structured in different ways.
Consider an image that is represented with RGB.
This can be represent with a Structure of Arrays (SoA) and a Array of Structures (AoS).
These two structures are presented in \cref{lst:soa aos}.

\begin{lstlisting}[caption={Example of SoA and AoS with RGB}, label={lst:soa aos}]
struct {
  int R, G, B;
} AoS[N];

struct {
  int R[N], G[N], B[N];
} SoA;
\end{lstlisting}

Coalesced memory access is where memory is access in a fashion that agrees with spatial lcality, and thus neighbouring data is accessed in the near future.
This type of memory access is desired in GPU because of the structure of the caches inside the cores.
The best suited structure is the SoA because the access to memory is done in a coalesced manner.
