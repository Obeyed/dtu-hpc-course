\section{Compact}
\label{sec:compact}

This section presents the algorithm commonly known as compact (also known as filter).
The idea is to take some input and only return the items that obey some predicate, e.g. whether or not the least significant bit (LSB) is 0.
The algorithm can be divided into three steps
%
\begin{enumerate}
  \item Calculate predicate array
  \item Calculate scatter addresses from predicate array
  \item Map desired items to output, given the scatter addresses
\end{enumerate}
%
\Cref{lst:predicate} briefly presents a simple computation whether or not the LSB is 0, and saves the result to the \ttt{predicate} array.

\begin{lstlisting}[numbers=none, caption={LSB equal to 0 -- save items' result to predicate array.}, label={lst:predicate}]
predicate[idx] = (int)((input[idx] & 1) == 0);
\end{lstlisting}

The next step is to compute the addresses whereto the given items from the input array must be moved to.
This can be computed with an exclusive sum scan over the \ttt{predicate} array.
\Cref{tab:excl sum scan} presents a simple example, where the top row presents the input indices, then the items in each index, the predicate for that item, and the scatter address based on the predicates, respectively.

\begin{table}[htb]
  \centering
  \begin{tabular}{r | c c c c c c}
    \toprule
    \tbf{idx}             & 0 & 1 & 2 & 3 & 4 & 5 \\
    \midrule
    \tbf{items}           & 4 & 5 & 6 & 7 & 8 & 9 \\
    \tbf{predicate}       & 1 & 0 & 1 & 0 & 1 & 0 \\
    \tbf{scatter address} & 0 & 1 & 1 & 2 & 2 & 2 \\
    \bottomrule
  \end{tabular}
  \caption{Predicate and scatter address output given the input}
  \label{tab:excl sum scan}
\end{table}

From the scatter addresses it is now possible to map the values where the predicate is 1 to a new output array.
The contents of that array is thus the values where the LSB is 0.
This array will be of length 3, because the reduction of the predicate array yields the value 3.
The array will be \ttt{[4, 6, 8]}.


