\section{Histogram}
\label{sec:histogram}

The histogram algorithm simply categorises each input element into a bin, for which it obeys the given category.
Such a category could be to have $k$ bins, and we want to categorise the elements by their remainder after being divided by $k$, i.e. $8\ \%\ 7 = 1$, where $k = 7$ and the element is placed in bin number $1$.
A serial example is presented int \cref{lst:hist seq}.
We will have $k$ bins in the output.

\begin{lstlisting}[caption={Serial histogram}, label={lst:hist seq}]
void histogram(int *h_in, int *h_out, int IN_SIZE, int k) {
  for (int l = 0; l < IN_SIZE; l++) 
    h_out[(h_in[l] % k)]++;
}
\end{lstlisting}

In \cref{sec:challenges with parallel programs} we presented a naive implementation of the histogram kernel.
To avoid race conditions we use the \ttt{atomicAdd()} function to perform an atomic update on the given bin.
We present the parallel implementation in \cref{lst:histo par}.

\begin{lstlisting}[caption={Simple parallel histogram implementation}, label={lst:histo par}]
__global__ 
void histo_kernel(int *d_bins, int *d_in, int BIN_SIZE, int SIZE) {
  int tid = threadIdx.x + blockDim.x * blockIdx.x;
  if (tid >= SIZE) return; // checking for out-of-bounds

  int bin = d_in[tid] % BIN_SIZE;
  atomicAdd(&(d_bins[bin]), 1);
}
\end{lstlisting}

