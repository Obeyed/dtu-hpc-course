\section{Debugging}
\label{sec:debugging and profiling}
To debug we use two approached; \ttt{printf} for testing the logic of our application and \ttt{cuda-memcheck} to catch cuda memory errors.

The \ttt{cuda-memcheck} is a command-line tool that allows print of captured cuda-errors doing runtime.
It can be used to find issues with memory access, thread ordering, race conditions and hardware reported program errors.
It is invoked as folllows
%
\begin{quote}
  \ttt{cuda-memcheck [options] application-name [application-options]}
\end{quote}
%
As with the \ttt{nvprof} from \cref{sec:introduction to analysis of optimisation} the \ttt{cuda-memcheck} tool has a variety of flags for the option input.
In the option input \ttt{memcheck} is sat by default.
We further tested the option flag \ttt{racecheck}.
The \ttt{racecheck} allows us to detect write-after-write hazards, where two or more threads attempt to update the same memory location simultaneously.
For example, our race condition from \cref{sec:challenges with parallel programs} gave the following stacktrace\todo{outprint}.~\cite{cudamemcheck2015doc}
